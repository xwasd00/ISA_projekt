\documentclass[a4paper, 12pt, titlepage]{article}
\usepackage{times}
\usepackage[utf8]{inputenc}
\usepackage[czech]{babel}
\usepackage[left=2cm, top=3cm, text={17cm, 24cm}]{geometry}
\usepackage[unicode]{hyperref}
\usepackage{graphicx}
\usepackage{float}
\urlstyle{same}
\renewcommand{\figurename}{Obrázek}

\begin{document}

\begin{titlepage}
\begin{center}
\textsc{\Huge
Vysoké učení technické v~Brně\\[0.3em]
\huge Fakulta informačních technologií}\\
\vspace{\stretch{0.382}}
{\LARGE Monitoring SSL spojení\\[0.3em]
\Huge Projekt do ISA}\\
\vspace{\stretch{0.618}}
\end{center}
{\Large \today \hfill Michal Sova}
\end{titlepage}

\tableofcontents
\newpage

\section{Úvod do problematiky}


%\begin{figure}[H]
%    \centering
%    \scalebox{0.39}{\includegraphics{ip-v4-datagram-header.png}}
%    \caption{Hlavička IPv4\cite{obr1}}
%    \label{obrazek 1}
%\end{figure}

\section{Implementace}
Program nejdříve zpracuje argumenty programu pomocí getopts\cite{getopt}. Otevře rozhraní pro zachytávání paketů nebo otevře soubor se síťovým provozem, nastaví filtr na zachytávání tcp paketů a pomocí funkce \verb|pcap_loop| a \verb|callback| funkce\cite{pcap} zachytává pakety. V~\verb|callback| funkci zavolá funkci \verb|getAddress|. Tato funkce zjistí z IP nebo IPv6 hlavičky zdrojovou a cílovou adresu a velikost IP hlavičky. Z TCP se získá zdrojový a cílový port. Následně se spojení vyhledá ve vektoru \verb|conn_vec|, v případě nenalezení se do vektoru vloží nové spojení. Následně se zjistí, zda-li jde o SSL spojení. V případě ukončení spojení se SSL spojení vypíše a spojení se smaže z vektoru spojení. Za ukončení spojení se považuje zachycení TCP s příznakem FIN z obou stran (od serveru i klienta), nebo zachycení TCP s příznakem RST z jedné strany.


\subsection{SSL}
Cyklus v \verb|callback| funkci přečítá bajt po bajtu obsah a hledá příznaky SSL hlavičky, která obsahuje typ hlavičky, verzi protokolu a délku zprávy. Toto řešení je z důvodu možné různé velikosti obsahu tcp a velikosti SSL obsahu. V případě handshake program sleduje, zda se jedná o Client Hello nebo Server Hello. V případě Client Hello se snaží najít Server Name Indication (SNI). V případě nenalezení se místo SNI nic nevypíše.


\subsection{využití knihoven}

Bylo využito knihovny pcap.h a jejich funkcí pro zachytávaní paketů. Knihovny netinet/ip.h pro použití struktury \verb|struct ip|, ze které analyzátor načítá adresy a kontroluje další hlavičku. Podobně díky knihovně netinet/ip6.h a struktuře \verb|struct ip6_hdr|. Z~knihovny netinet/tcp.h bylo využito struktury \verb|struct tcphdr|. 

\section{Základní informace o programu}
Informace o SSL spojení se vypisují ve tvaru: \verb|<timestamp>,<client~ip>,<client~port>,<server~ip>,<SNI>,<bytes>,<packets>,<duration sec>|\newline
Všechny časové údaje jsou zaokrouhlené na 6 desetinných míst.

\subsection{Návod na použití}
\begin{itemize}
  \item \verb|./sslsniff -h| pro zobrazení nápovědy.
  \item \verb|./sslsniff -r file.pcapng| kde file.pcapng obsahuje zachycený síťový provoz.
  \item \verb|./sslsniff -i interface| kde interface je rozhraní pro zachytávání. Tento příkaz je nutný zadat s dostatečnými právy (sudo).
  \item Není-li uveden interface (avšak parametr -i je přítomen), vypíše se seznam aktivních rozhraní.
\end{itemize}

\subsection{Upřesnění chování}
\begin{itemize}
    \item Program ignoruje neznámé argumenty.
    \item V případě zadání obou argumentů ('-r' i '-i'), program upřednostní čtení ze souboru (argument -r).
    \item Není-li ani jeden z parametrů `-r` a `-i` uveden, vypíše se nápověda. 
\end{itemize}

\newpage
\section{Použité zdroje}
\begin{thebibliography}{99}
\bibitem{getopt} Ashwin, V., 2015. How To Parse Program Options In C++ Using Getopt\_Long. [online] Code Yarns. Dostupné z:\url{https://codeyarns.com/2015/01/30/how-to-parse-program-options-in-c-using-getopt\_long/}
\bibitem{pcap} Carstens, T., 2020. Programming With Pcaptcpdump/LIBPCAP Public Repository. [online] Tcpdump.org. Dostupné z: \url{https://www.tcpdump.org/pcap.html}

\end{thebibliography}

\subsection{Obrázky}
\begin{thebibliography}{99}
% \bibitem{obr1} GeeksforGeeks. n.d. Introduction And Ipv4 Datagram Header - Geeksforgeeks. [online] Dostupné z: \url{https://www.geeksforgeeks.org/introduction-and-ipv4-datagram-header/}
\end{thebibliography}


\end{document}
